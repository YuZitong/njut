\documentclass[cs4size,a4paper,fancyhdr,fntef]{ctexbook}
\usepackage{amsmath,amsthm,amsfonts,amssymb,bm}
\usepackage{hyperref}
\usepackage[sort&compress,numbers]{natbib}
\usepackage{graphicx}
\usepackage{graphicx}
\usepackage{geometry}
\usepackage{enumitem}
\usepackage{times}
\usepackage{fancyvrb}
\usepackage{titletoc}


\geometry{top=3.5cm,bottom=3.5cm,left=3.2cm,right=3.2cm}
\geometry{headheight=2.6cm,headsep=3mm,footskip=13mm}

\parskip 0.5ex plus 0.25ex minus 0.25ex

\def\cleardoublepage{\clearpage\if@twoside \ifodd\c@page\else
 \thispagestyle{empty}%
 \hbox{}\newpage\if@twocolumn\hbox{}\newpage\fi\fi\fi}

\makeatletter
\hypersetup{CJKbookmarks,%
    bookmarksnumbered,%
        colorlinks,%
        linkcolor=black,%
        citecolor=black,%
        plainpages=false,%
        pdfstartview=FitH}


\renewcommand{\floatpagefraction}{0.80}
\newcommand\NJUTspace{\protect\CTEX@spaceChar\protect\CTEX@spaceChar}
\def\CTEX@contentsname{\zihao{3}\songti\bfseries 目\NJUTspace 录}
\def\CTEX@listfigurename{插\NJUTspace 图}
\def\CTEX@listtablename{表\NJUTspace 格}
%footnote
\fancypagestyle{plain}{%
    \fancyhf{}
    \fancyhead[C]{\if@mainmatter \small \leftmark\fi}
    \fancyfoot[C]{\small ~\thepage~}
    \renewcommand{\headrulewidth}{\if@mainmatter 0.7pt\else 0pt \fi}
}
\pagestyle{plain}

%underline
\def\NJUT@underline[#1]#2{%
    \CTEXunderline{\hbox to #1{\hfill#2\hfill}}}
\def\NJUTunderline{\@ifnextchar[\NJUT@underline\CTEXunderline}

%value
\def\NJUT@value@title{论文题目}
\def\NJUT@value@author{}
\def\NJUT@value@advisor{}
\def\NJUT@value@major{}
\def\NJUT@value@department{}
\def\NJUT@value@grade{}
\def\NJUT@value@englishtitle{}
\def\NJUT@value@englishauthor{}
\def\NJUT@value@englishadvisor{}
\def\NJUT@value@englishmajor{}
\def\NJUT@value@englishdepartment{}
\def\NJUT@value@englishgrade{}

%abstract
\def\NJUT@abs@label@bar{南京大学本科生毕业论文(设计)中文摘要}
\def\NJUT@abs@label@englishbar{南京大学本科生毕业论文(设计)英文摘要}
\def\NJUT@abs@label@title{毕业论文题目:}
\def\NJUT@abs@label@major{专业}
\def\NJUT@abs@label@department{院系}
\def\NJUT@abs@label@author{级本科生姓名:}
\def\NJUT@abs@label@advisor{指导教师 (姓名、职称):}
\def\NJUT@abs@label@abstract{摘要:}
\def\NJUT@abs@label@keywords{关键词}
\def\NJUT@abs@label@englishabstract{ABSTRACT:}
\def\NJUT@abs@label@englishkeywords{KEYWORDS:~}

\newenvironment{abstract}{
    \pdfbookmark[0]{\NJUT@abs@label@abstract}{abstract}
    \begin{center}
        {\bf\kaishu\zihao{-2} \uuline{\NJUT@abs@label@bar}}
    \end{center}

    {\kaishu\zihao{4}%
        \noindent \NJUT@abs@label@title \NJUTunderline[315pt]{\NJUT@value@title}

        \noindent \NJUTunderline[400pt]{}

        \noindent \NJUTunderline[85pt]{\NJUT@value@department}\NJUT@abs@label@department%
        \NJUTunderline[71pt]{\NJUT@value@major}\NJUT@abs@label@major%
        \NJUTunderline[43pt]{\NJUT@value@grade}\NJUT@abs@label@author%
        \NJUTunderline[60pt]{\NJUT@value@author}

        \noindent \NJUT@abs@label@advisor \NJUTunderline[252pt]{\NJUT@value@advisor}

        \NJUT@abs@label@abstract

    }
}{}
\newcommand\keywords[1]{\vspace{2ex}\noindent{\kaishu \NJUT@abs@label@keywords} #1}

\newenvironment{englishabstract}{%
    \clearpage
    \begin{center}
        {\bf\kaishu\zihao{-2} \uuline{\NJUT@abs@label@englishbar}}
    \end{center}
    {\zihao{4}%
    \begin{description}[font=\normalfont,leftmargin=4em]
        \item[THESIS:]\NJUT@value@englishtitle
        \item[DEPARTMENT:]\NJUT@value@englishdepartment
        \item[SPECIALIZATION:]\NJUT@value@englishmajor
        \item[UNDERGRADUATE:]\NJUT@value@englishgrade
        \item[MENTOR:]\NJUT@value@englishadvisor
        \item[ABSTRACT:]\hfil
    \end{description}
    }
}{}
\newcommand\englishkeywords[1]{%
    \vspace{2ex}\noindent{\NJUT@abs@label@englishkeywords} #1}

%%
%contents
%%

\newcommand\NJUTtableofcontents{%
\@mainmattertrue
\addcontentsline{toc}{chapter}{目\NJUTspace 录}
\@mainmatterfalse
\tableofcontents
}
\addtocontents{toc}{\let\string\CTEX@spaceChar\relax}

\titlecontents{chapter}[2em]
              {\addvspace{0.6pc}\heiti\bfseries\zihao{4}}
              {\thecontentslabel\enspace}
              {}
              {\titlerule*[0.5em]{$\cdot$}\contentspage}
              [\addvspace{0.25pc}]

\titlecontents{section}[4em]
              {\addvspace{0.1pc}\songti\zihao{-4}}
              {\thecontentslabel\enspace}
              {}
              {\titlerule*[0.5em]{$\cdot$}\contentspage}
              [\addvspace{0.25pc}]

\newcommand\Nchapter[1]{%
    \if@mainmatter%
        \@mainmatterfalse%
        \chapter{#1}%
        \@mainmatterture%
    \else
        \chapter{#1}%
    \fi}
%bib
\renewcommand\bibfont{\zihao{-4}}
\def\CTEX@references{\zihao{1}参考}

%%section
\setcounter{secnumdepth}{4}
\def\CTEX@chapter@nameformat{\bfseries\heiti\zihao{4}}
\def\CTEX@chapter@titleformat{\bfseries\heiti\zihao{4}}
\def\CTEX@chapter@beforeskip{15\p@}
\def\CTEX@chapter@afterskip{12\p@}
\def\CTEX@section@format{\bfseries\heiti\zihao{4}\centering}
\def\CTEX@section@beforeskip{-3ex \@plus -1ex \@minus -.2ex}
\def\CTEX@section@afterskip{1.0ex \@plus .2ex}
\def\CTEX@subsection@format{\bfseries\heiti\zihao{-4}}
\def\CTEX@subsection@indent{2\ccwd}
\def\CTEX@subsection@beforeskip{-2.5ex \@plus -1ex \@minus -.2ex}
\def\CTEX@subsection@afterskip{1.0ex \@plus .2ex}
\def\CTEX@subsubsection@format{\bfseries\heiti\zihao{-4}}
\def\CTEX@subsubsection@indent{2\ccwd}
\def\CTEX@subsubsection@beforeskip{-2ex \@plus -1ex \@minus -.2ex}
\def\CTEX@subsubsection@afterskip{1.0ex \@plus .2ex}



\newtheoremstyle{break}% name
{}% Space above, empty = `usual value'
{}% Space below
{\itshape}% Body font
{}% Indent amount (empty = no indent, \parindent = para indent)
{\bfseries}% Thm head font
{.}% Punctuation after thm head
{\newline}% Space after thm head: \newline = linebreak
{}% Thm head spec


%%
%% the theorems definitions
%%
\theoremstyle{plain}
\newtheorem{algo}{算法~}[chapter]
\newtheorem{thm}{定理~}[chapter]
\newtheorem{lem}[thm]{引理~}
\newtheorem{prop}[thm]{命题~}
\newtheorem{cor}[thm]{推论~}
\theoremstyle{definition}
\newtheorem{defn}{定义~}[chapter]
\newtheorem{conj}{猜想~}[chapter]
\newtheorem{exmp}{例~}[chapter]
\newtheorem{rem}{注~}
\newtheorem{case}{情形~}
\theoremstyle{break}
\newtheorem{bthm}[thm]{定理~}
\newtheorem{blem}[thm]{引理~}
\newtheorem{bprop}[thm]{命题~}
\newtheorem{bcor}[thm]{推论~}
\renewcommand{\proofname}{{\bf 证明}} 